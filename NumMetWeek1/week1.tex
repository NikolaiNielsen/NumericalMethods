% Nikolai Nielsens "Fysiske Fag" preamble
\documentclass[a4paper,10pt]{article} 	% A4 papir, 10pt størrelse
\usepackage[english]{babel}
\usepackage{Nikolai} 					% Min hjemmelavede pakke
\usepackage[dvipsnames]{xcolor}
\usepackage{gensymb}

% Margen
\usepackage[margin=1in]{geometry}

% Max antal kolonner i en matrix. Default er 10
%\setcounter{MaxMatrixCols}{20}

% Hvor dybt skal kapitler labeles?
%\setcounter{secnumdepth}{4}	
%\setcounter{tocdepth}{4}


% Hvilket nummer skal der startes med i sections? (n-1)
%\setcounter{section}{0}	

% Til nummerering af ligninger. Så der står (afsnit.ligning) og ikke bare (ligning)
\numberwithin{equation}{section}


% Header
%\usepackage{fancyhdr}
%\head{}
%\pagestyle{fancy}

%Titel
\title{Numerical Methods in Physics Week 1}
\author{Nikolai Plambech Nielsen, LPK331. Version 1.0}
\date{\today}

\begin{document}
	\maketitle
	\section{Nuclear Decay}
	We have two radioactive isotopes, A and B, with populations $ N_A(t) $ and $ N_B(t) $, and decay times $ \tau_A $  and $ \tau_B $. Type A decays into type B, while B decays into something else we don't track. The relevant differential equations are
	\begin{equation}\label{key}
		\diff[\ud]{N_A}{t} = - \frac{N_A}{\tau_A}, \quad \diff[\ud]{N_B}{t} = \frac{N_A}{\tau_A} - \frac{N_B}{\tau_B}.
	\end{equation}
	The purpose of this assignment is to solve this problem numerically for a number of different conditions, using Euler integration. This method is used for its simplicity and ease of implementation. For all of the following simulations and plots, the values $ \Delta t = 0.1 $ is used.
	
	\subsection{Compare the numerical and analytical solutions}
	The analytical solutions are given in the assignment, and are as follows:
	\begin{align}
		N_A(t) &= N_A(0) \exp\pp{-\frac{t}{\tau_A}}, \\
		N_B(t) &= \begin{cases}
		N_B(0) \exp\pp{-\frac{t}{\tau_A}} + t \frac{N_A(0)}{\tau_A} \exp\pp{-\frac{t}{\tau_A}}, & \tau_A= \tau_B, \\
		N_B(0) \exp\pp{-\frac{t}{\tau_B}} + \frac{N_A(0)}{\frac{\tau_A}{\tau_B} - 1} \bb{\exp\pp{-\frac{t}{\tau_A}} - \exp\pp{-\frac{t}{\tau_B}}}, & \tau_A \neq \tau_B.
		\end{cases}
	\end{align}	
	The results, along with residual plots, for $ N_A(0) = N_B(0) = 1000 $ and $ \tau_A = 5, \tau_B = 10 $ are shown below, with the analytical solution for $ \tau_A \neq \tau_B $:
	\begin{figure}[H]
		\centering
		\includegraphics[width=0.7\linewidth]{unequaltau.pdf}
		\caption{\textbf{INSERT PURDY CAPTION PL0X}}
		\label{fig:unequalTau}
	\end{figure}
\textbf{COMMENT ON RESULT, YAS, V V V GUT}
	To demonstrate the solution for $ \tau_A = \tau_B $, the initial conditions of $ N_A=N_B = 1000 $ and $ \tau_A=\tau_B = 10 $ are shown below. Again with accompanying residual plots:
	\begin{figure}[H]
		\centering
		\includegraphics[width=0.7\linewidth]{equaltau.pdf}
		\caption{\textbf{INSERT PURDY CAPTION PL0X}}
		\label{fig:equalTau}
	\end{figure}
	
	\subsection{Explain the limit of $ \tau_A/\tau_B \gg 1 $}
	In this limit, the decay of type $ B $ is much faster than that of type $ A $. This means, that on any appreciable time scale of change for $ N_A $, the population of $ B $ reaches a steady-state solution, where $ \ud N_B/\ud t = 0 $. As such the differential equation for $ N_B $ becomes
	\begin{equation}\label{eq:steadyNB}
		\diff[\ud]{N_B}{t} = \frac{N_A}{\tau_A} - \frac{N_B}{\tau_B} = 0, \quad \Rightarrow \quad N_B(t) = N_A \frac{\tau_B}{\tau_A}.
	\end{equation}
	This is also seen in a simulation, where $ \tau_A = 3600,\tau_B = 5 $:
	\begin{figure}[H]
		\centering
		\includegraphics[width=0.7\linewidth]{largetau.pdf}
		\caption{\textbf{INSERT PURDY CAPTION PL0X}}
		\label{fig:largeTau}
	\end{figure}
	as seen, the steady-state solution for $ N_B $ (\eqref{eq:steadyNB}) does correctly predict the behaviour for $ N_B $, when this is approximately 0. This makes sense; the change in $ N_A $ is almost constant, corresponding to when $ N_B $ is approximately 0, which is the steady state for this particular problem.
	
	
	\section{Projectile motion}
	For this assignment, projectile motion is to be simulated, again using the Euler-method. This gives another example of how to use this simple method of simulation, for a problem which (without wind resistance) has an analytical solution. After this, uncharted waters are encountered, when wind resistance is included. Here no analytical solution is given, and all reliance upon the previous methods, like residuals between the numerical and analytical solutions, are not applicable.
	
	Furthermore, for this problem, a second derivative is used, whereby a second Euler integration is needed, to complete the time step. The relevant differential equations are
	\begin{equation}\label{key}
		\diff[\ud]{\V{r}}{t} = \V{v}, \quad \diff[\ud]{\V{v}}{t} = -g\Vy.
	\end{equation}
	where the analytical solution is computed by integrating the second equation twice, and using the relevant initial conditions:
	\begin{equation}\label{key}
		\V{r}(t) = \begin{pmatrix}
		x(t) \\ y(t)
		\end{pmatrix} = \begin{pmatrix}
		x_0+v_{x,0} t \\ y_0+v_{y,0}t-gt^2/2
		\end{pmatrix}. 
	\end{equation}
	
	
	\subsection{Without wind resistance}
	The numerical and analytical solutions to projectile motion, given the initial conditions of $ x_0 = 0,y_0 = 2 \e{m},v_0 = 4 \e{m/s}$ and $ \theta = 70\degree$, where $ v_{x,0} = v_0 \cos \theta $ and $ v_{y,0} = v_0 \sin \theta $. For the numerical solution a value of $ \Delta t = 0.01 \e{s}$ is used. The results are shown below, along with the absolute residual, as a function of time:
	\begin{figure}[H]
		\centering
		\includegraphics[width=0.7\linewidth]{projectile.pdf}
		\caption{\textbf{INSERT PURDY CAPTION PL0X}}
		\label{fig:projectile}
	\end{figure}
	the absolute residual is also the global truncation error, as defined in the slides for the week 1 lectures. As is expected, the error increases linearly in time, given the linear increase in velocity. This means that the Euler integration underestimates the velocity by a constant amount per time step, leading to a linear increase in global truncation error.
	
	
	Next the effect of the size of the time step $ \Delta t $ is analysed. This is done by creating a vector of logarithmically spaced values for $ \Delta t $, between $ 10^{-1} \e{s} $ and $ 10^{-8} \e{s}$, running the simulation, and recording the global truncation error. If any smaller step size is to be investigated, the simulation as to be broken into pieces, as the size of the arrays exceed the amount of memory available on the desktop workstation used.
	
	This could be done by recording the first $ 10^6 $ steps, storing the final results, clearing the arrays from memory and then running the simulation again, with the final results as the new initial conditions, until the full simulation has been completed.
	
	However, with the step sizes as mentioned, the final global truncation error becomes
	\begin{figure}[H]
		\centering
		\includegraphics[width=0.7\linewidth]{projError.pdf}
		\caption{\textbf{INSERT PURDY CAPTION PL0X}}
		\label{fig:projError}
	\end{figure}
	On the double-log plot, the relation between global truncation error and time step size is definitely polynomial, as the graph is that of a straight line. The slope then corresponds to the polynomial degree, and as this is approximately 1 \textbf{REMEMBER CHECKUP}, the relation is taken as being linear.
	
	\subsection{With wind resistance}
	Next a drag force, quadratic in velocity, is introduced to the problem. The formula is
	\begin{equation}\label{key}
		\V{F}_d = -\frac{1}{2}C_D \rho A v \V{v}
 	\end{equation} 
 	where $ C_D $ is the coefficient of drag for the projectile, $ \rho $ is the density of the surrounding media, $ A $ is the cross sectional area of the projectile, and $ v $ is the magnitude of the velocity vector $ \V{v} $. As seen, this is directed opposite to the velocity. The differential equations become:
 	\begin{equation}\label{key}
 			\diff[\ud]{\V{r}}{t} = \V{v}, \quad \diff[\ud]{\V{v}}{t} = -\frac{1}{2m}C_D \rho A v \V{v} -g\Vy.
 	\end{equation}
 	where now the second differential equation is no longer linear, owing to the quadratic term in velocity.
 	
 	With no analytical solution given, other means for estimating the error has to be utilized. As confirmed in the example with no wind resistance, the global truncation error is linearly related to the size of the time step $ \Delta t $. As such, the global truncation error on the simulation with wind resistance will also approach 0 as the size of $ \Delta t $ is decreased. Given this, a method of attaining a desired level of precision is needed.
 	
 	For the problem at hand, the simulation is run with 100 logarithmically spaced values between $ 10^{-1} \e{s}$ and $ 10^{-8} \e{s} $. The final point of impact is logged, along with the time taken to run the simulation. The results are shown below on semilog and double-log plots respectively:
 	\begin{figure}[H]
 		\centering
 		\includegraphics[width = 0.7\linewidth]{simtime.pdf}
 		\caption{\textbf{INSERT PURDY CAPTION PL0X}}
 		\label{fig:simtime}
 	\end{figure}
 	the reason for also recording the simulation time is that while a simulation taking a week to run might be the most precise result achievable, it is not very practical, and might not be that much more precise than that of a simulation that only takes a minute or two. For example, the final simulation, with $ \Delta t = 10^{-8} \e{s} $, took just under 7 minutes. The simulation with the smallest $ \Delta t $, coming within 1\% of the result of the final simulation took just 0.0061 seconds, with a time step size of $ \Delta t = 0.0043 s $. So while the precision is within 1 \% of the final simulation, the run time is almost 5 orders of magnitude smaller. The desired precision can of course be chosen as needed, achieving a precise enough results whilst keeping run time small enough to be practical.
 	
 	
 	\section{Orbits of comets}
 	For this assignment, the trajectory of particles in a gravitational field is calculated both using Euler integration and 4th order Runge-Kutta integration, which uses a weighted average of several derivatives with smaller time steps, to calculate the final value of the next time step. The differential equations for this problem is
 	\begin{equation}\label{key}
 		\diff[\ud]{\V{r}}{t} = \V{v}, \quad \diff[\ud]{\V{v}}{t} = \frac{GM}{r^2}\U{r} = \frac{GM}{r^3}\V{r},
 	\end{equation}
 	where, for the purpose of simplicity, $ GM $ is set to $ 4\pi^2 \e{(AU)}^3/\e{(yr)}^3 $. This gives a circular orbit for an initial radius of $ 1 \e{AU} $ and initial tangential velocity of $ 2 \pi \e{AU}/\e{yr} $. Furthermore, the mass of the comet in orbit, is set to unity. 
 	
 	\subsection{Euler and RK4}
 	First, the two methods are compared for a circular orbit with a step size of $ \Delta t = 0.02 \e{yr} $ and a total simulation time of $ 1 \e{yr} $. The trajectories are shown in the figure below:
 	
	\begin{figure}[H]
		\centering
		\includegraphics[width=0.7\linewidth]{CircOrbit.pdf}
		\caption{\textbf{INSERT PURDY TEXT}}
		\label{fig:CircOrbit}
	\end{figure}
 	As seen, neither of the simulations actually complete an orbit, however the Runge-Kutta method comes significantly closer. To quantify this, the following fractional error is calculated:
 	\begin{equation}\label{key}
 		\text{err} = \frac{|\V{r}_f-\V{r}_0|}{|\V{r}_0|}
 	\end{equation}
 	where $ \V{r}_0 $ is the initial and $ \V{r}_f $ is the final position of the comet. Next, the simulation is run for 60 different values of $ \Delta t $, logarithmically spaced between $ 10^{-1} \e{yr} $ and $ 10^{-6} \e{yr} $:
	\begin{figure}[H]
		\centering
		\includegraphics[width=0.7\linewidth]{KeplerdtErr.pdf}
		\caption{\textbf{INSERT PURDY TEXT}}
		\label{fig:KeplerdtErr}
	\end{figure}
	As seen from the figure, the Runge-Kutta method consistently outperforms the Euler method in fractional error, though the error in the Runge-Kutta method fluctuates, sometimes over an order of magnitude. Rounded up, the Runge-Kutta needs a $ \Delta t $ of the order $ 10^{-3} \e{yr} $ to have a fractional error lower than 1\%, whilst the Euler method needs one in the order of $ 10^{-5} \e{yr} $, 2 orders of magnitude lower.
	
	\subsection{RK4 and the analytical solution}
	
	
\end{document}

